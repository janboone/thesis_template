% Created 2022-02-26 Sat 17:35
% Intended LaTeX compiler: pdflatex
\documentclass[11pt]{article}
\usepackage[utf8]{inputenc}
\usepackage[T1]{fontenc}
\usepackage{graphicx}
\usepackage{grffile}
\usepackage{longtable}
\usepackage{wrapfig}
\usepackage{rotating}
\usepackage[normalem]{ulem}
\usepackage{amsmath}
\usepackage{textcomp}
\usepackage{amssymb}
\usepackage{capt-of}
\usepackage{hyperref}
\makeatletter
\newcommand{\citeprocitem}[2]{\hyper@linkstart{cite}{citeproc_bib_item_#1}#2\hyper@linkend}
\makeatother
\author{Adam Smith}
\date{\today}
\title{Thesis title}
\hypersetup{
 pdfauthor={Adam Smith},
 pdftitle={Thesis title},
 pdfkeywords={},
 pdfsubject={},
 pdfcreator={Emacs 27.2 (Org mode 9.4.4)}, 
 pdflang={English}}
\begin{document}

\maketitle
\begin{abstract}
This document shows you the syntax to type your thesis in latex or org-mode. It illustrates how to make footnotes, tables, equations and references to tables, equations etc.

If you want to work with latex only, look at the \texttt{.tex} file of this document. If you want to use emacs org-mode, then use the \texttt{.org} file. The pdf shows what the file looks like if you export it.

Running python code in this file only works in emacs org-mode; not in latex.
\end{abstract}

\newpage



\setcounter{tocdepth}{2}
\tableofcontents


\section{Introduction}
\label{sec:org75302b8}
\label{sec:intro}

This file shows you how to use \href{https://www.gnu.org/software/emacs/}{emacs} \href{https://orgmode.org/}{org mode} to write a thesis. Shows you how to cite references, make footnotes, equations etc.

Alternatively, you can use latex directly in which case you can consider the file in this repository that ends in \texttt{.tex}.

In order to use emacs, you need to install it. The official website's download information: \url{https://www.gnu.org/software/emacs/download.html}

In Section \ref{sec:intro} we explain how emacs can be installed.


\section{Literature references}
\label{sec:org5dd2a3d}

There are a number of ways in which you can do literature citations in org-mode. We will work with org-ref:
\begin{itemize}
\item \url{https://github.com/jkitchin/org-ref}
\end{itemize}

The syntax for including references is as follows. See Farrell and Klemperer \citeprocitem{2}{2007} for an analysis. We can also have references between brackets (\citeprocitem{1}{Athey and Imbens 2019}): that is, \texttt{citep} instead of \texttt{citet}. \texttt{armstrong-2007-chapt-coord} is the bibtex key, as you can see in the file \texttt{references.bib}.

If you use the \texttt{init.el} file for emacs, you can use the keys: C-c ] (press control (Ctrl) and c together; release these keys and then press the ]-key). The bottom part of the screen then gives you possible papers to cite from your \texttt{references.bib} file.


\section{Model}
\label{sec:org867adc8}

As we explained in section \ref{sec:intro}.

Here we have some in-line math: \(x^2\).\footnote{This is a footnote.}

\begin{equation}
\label{eq:1}
a^2 + b^2 = c^2
\end{equation}

As we show in equation \eqref{eq:1}.

See Table \ref{table1}.

\begin{table}[htbp]
\caption{\label{tab:org35712fb}\label{table1} This table shows unemployment and gdp per head.}
\centering
\begin{tabular}{lrr}
country & unemployment & gdp\\
\hline
NL & 0.06 & 20000\\
UK & 0.01 & 19500\\
BE & 0.08 & 21100\\
\hline
average & 0.05 & 20200\\
\end{tabular}
\end{table}

\begin{figure}[htbp]
\centering
\includegraphics[width=.9\linewidth]{./fig.png}
\caption{\label{fig:org1a839e9}\label{figure1} Figure with unemployment and gdp}
\end{figure}

See Figure \ref{figure1}.


\section{Conclusion}
\label{sec:org488ea8a}



\section{Bibliography}
\label{sec:org56e9727}



\hypertarget{citeproc_bib_item_1}{Athey, Susan, and Guido W. Imbens. 2019. “Machine Learning Methods That Economists Should Know About.” \textit{Annual Review of Economics} 11 (1): 685–725. doi:\href{https://doi.org/10.1146/annurev-economics-080217-053433}{10.1146/annurev-economics-080217-053433}.}

\hypertarget{citeproc_bib_item_2}{Farrell, Joseph, and Paul Klemperer. 2007. “Chapter 31 Coordination and Lock-in: Competition with Switching Costs and Network Effects,” edited by M. Armstrong and R. Porter, 3:1967–2072. Handbook of Industrial Organization, Supplement C. Elsevier. doi:\href{https://doi.org/10.1016/S1573-448X(06)03031-7}{10.1016/S1573-448X(06)03031-7}.}





\newpage
\appendix


\section{Things to install}
\label{sec:orgea6af2f}
\label{sec:install}

\subsection{latex}
\label{sec:orgd31fa40}

Install latex: \url{https://www.latex-project.org/get/}



\subsection{latex editor}
\label{sec:org97b6068}

\begin{itemize}
\item winedt: \url{https://www.winedt.com/}
\item overleaf: \url{https://www.overleaf.com/}
\item texmaker: \url{https://www.xm1math.net/texmaker/}
\item tex studio: \url{https://www.texstudio.org/}
\end{itemize}

More general editors where you can also edit latex:

\begin{itemize}
\item atom: \url{https://atom.io/}
\begin{itemize}
\item and how to use with latex: \url{https://towardsdatascience.com/setting-up-latex-on-your-atom-editor-7ea624571d50}
\end{itemize}
\item vim: \url{https://www.vim.org/docs.php}
\end{itemize}



\subsection{git}
\label{sec:orgb0cccdd}

install git: \url{https://git-scm.com/downloads}

\subsection{Emacs on Windows}
\label{sec:org8ef348d}

\begin{itemize}
\item go to: \url{http://mirror.team-cymru.com/gnu/emacs/windows/emacs-27/}
\item download emacs-27.2-x86\textsubscript{64}-installer.exe to your Downloads folder: \url{http://mirror.team-cymru.com/gnu/emacs/windows/emacs-27/emacs-27.2-x86\_64-installer.exe}
\item run the downloaded \texttt{exe} file
\end{itemize}

\subsection{Emacs on Mac OS}
\label{sec:org091dc2d}

For Mac Os:
\begin{itemize}
\item install homebrew: \url{https://brew.sh/}
\end{itemize}

Open a terminal and type the following lines:

\begin{verbatim}
brew tap d12frosted/emacs-plus
brew install emacs-plus
\end{verbatim}

\subsection{Emacs on Linux}
\label{sec:org3ddf458}

When you are using Linux, you probably know what you are doing. But just in case, the commands for your package manager can be found here: \url{https://www.gnu.org/software/emacs/download.html}



\subsection{org-mode}
\label{sec:org903ce05}

When you install emacs, org-mode is installed as well (comes with emacs)


\subsection{introductions to emacs}
\label{sec:orgae425b7}

It is easy to get lost in emacs. Hence do not try to use everything at once. A couple of basic things, you need from the start (like opening and saving files). For the other things: move step-by-step. 

A great starting point, explaining key-bindings etc. is:
\begin{itemize}
\item \url{https://systemcrafters.net/emacs-essentials/absolute-beginners-guide-to-emacs/}
\begin{itemize}
\item and the video that goes with it: \url{https://www.youtube.com/watch?v=48JlgiBpw\_I}
\item this explains things like "M-x", "C-c", "C-x" etc. which you can see when you use menu items like "file"
\begin{itemize}
\item to illustrate, use your mouse to click on "File" in the top left corner
\item the first item is: "Visit New File\ldots{} C-x C-f"
\item you can click on this item to open a file; but you can also use the key combination C-x C-f which means: press Control (Ctrl) and x together; release these keys; then press Ctrl and f together. This allows you to open a file. If you type the name of a file that does not exist yet, this new file will be created
\item you save a file with C-x C-s; hence you can quickly save a file by pressing these keys without having to reach for the mouse
\item the emacs configuration below helps as it uses the which-key package. After typing C-x, it shows you what other keys you can use.
\end{itemize}
\end{itemize}
\end{itemize}


\url{https://www.youtube.com/playlist?list=PL9KxKa8NpFxIcNQa9js7dQQIHc81b0-Xg}






\subsection{next steps}
\label{sec:org873adc7}

\subsubsection{scimax}
\label{sec:orgc1042b1}

\url{https://github.com/jkitchin/scimax}

youtube playlist with scimax features: \url{https://www.youtube.com/playlist?list=PL0sMmOaE\_gs3E0OjExoI7vlCAVygj6S4I}

\subsubsection{Doom}
\label{sec:org02ee11b}

\url{https://github.com/hlissner/doom-emacs}

Doom emacs for noobs: \url{https://www.youtube.com/watch?v=iab2z21cRqA}

Doom emacs getting started: \url{https://www.youtube.com/watch?v=dr\_iBj91eeI}


youtube playlist: \url{https://www.youtube.com/playlist?list=PLhXZp00uXBk4np17N39WvB80zgxlZfVwj}
\end{document}